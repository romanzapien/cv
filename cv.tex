%%%%%%%%%%%%%%%%%%%%%%%%%%%%%%%%%%%%%%%%%
% Friggeri Resume/CV
% XeLaTeX Template
% Version 1.1 (9/2/15)
%
% This template has been downloaded from:
% http://www.LaTeXTemplates.com
%
% Original author:
% Adrien Friggeri (adrien@friggeri.net)
% https://github.com/afriggeri/CV
%
% License:
% CC BY-NC-SA 3.0 (http://creativecommons.org/licenses/by-nc-sa/3.0/)
%
% Important notes:
% This template needs to be compiled with XeLaTeX and the bibliography, if used,
% needs to be compiled with biber rather than bibtex.
%
%%%%%%%%%%%%%%%%%%%%%%%%%%%%%%%%%%%%%%%%%

\documentclass[]{friggeri-cv} % Add 'print' as an option into the square bracket to remove colors from this template for printing

\addbibresource{bibliography.bib} % Specify the bibliography file to include publications

\begin{document}

\header{Román }{Zapién-Campos}{MSCA \& UKRI PostDoctoral Fellow @ University College London} % Your name and current job title/field

%----------------------------------------------------------------------------------------
%	SIDEBAR SECTION
%----------------------------------------------------------------------------------------

\begin{aside} % In the aside, each new line forces a line break
\section{contact}
%Rautenberg-Straße 2A
%24306 Plön, Germany
%~
%+44  (0)7746351077
%+52 5574040142
%~
\href{https://romanzapien.github.io}{romanzapien.github.io}
\href{mailto:roman.zapien@ucl.ac.uk}{roman.zapien@ucl.ac.uk}
\href{https://orcid.org/0000-0003-0221-0936}{ORCiD}
\href{https://www.linkedin.com/in/romanzapien-campos/}{LinkedIn}
\section{nationality}
Mexican
\section{languages}
Spanish - native
English - fluent (C2)
%TOEFL iBT 105/120
%IELTS 7/9
\section{programming}
python, R, bash \& C++ 
mathematica \& octave
\section{research interests}
ecology \& evolution
theoretical modelling
quantitative biology
microbiology
\end{aside}

%----------------------------------------------------------------------------------------
%	ACADEMIC EXPERIENCE SECTION
%----------------------------------------------------------------------------------------

\section{education (*) \& research experience}

\begin{entrylist}
\entry
{2023 -- }
{}
{University College London}
{London, UK}
{{\normalsize\emph{PostDoc} - Centre for Life's Origins and Evolution}}
{Quantitative ecology and evolution of microbial communities \\ \emph{Advisor:}  Prof Wenying Shou}
%------------------------------------------------
\entry
{2022 -- 2023 }
{(15 months)}
{Max Planck Institute for Evolutionary Biology}
{Plön, Germany}
{{\normalsize\emph{PostDoc}}}
{Bayesian inference of interactions from microbiome datasets \\ \emph{Advisor:}  Prof Arne Traulsen}
%------------------------------------------------
\entry
{2018 -- 2021}
{}
{* Max Planck Institute for Evolutionary Biology}
{Plön, Germany}
{{\normalsize\emph{PhD {\normalfont IMPRS for Evolutionary Biology}}}}
{\emph{Thesis:} Modelling the ecology of host-associated microbiomes \\ \emph{Advisor:}  Prof Arne Traulsen \\ \emph{Summary:} we developed models to study hosts and their microbiomes. Topics included: the colonization of hosts by microbes, parental inheritance of microbes, and the effect of differences in selection. Methods included: stochastic and differential equations. }%\\ \** Included two 6-weeks lab interships (09/2017 -- 12/2017). Firstly, in Prof Traulsen's group and secondly, in Prof Hinrich Schulenburg's in CAU (Kiel, Germany).}
%------------------------------------------------
\entry
{2015 -- 2016}
{}
{* University College London}
{London, UK}
{{\normalsize\emph{MRes {\normalfont Biosciences - Computational Biology}}} }
{\emph{Coursework:} seminars, scientific literature and skills, math. ecology \& research project. \\
\emph{Thesis:} The evolution of Hutchinson’s niche \& mutualism \\ \emph{Advisor:} Dr David Murrell \\ \emph{Summary:} we developed a spatial model to study the evolution of mutualism, finding that public goods could make communities more resilient to eco-evolutionary hazards.}
%------------------------------------------------
\entry
{2015 \& 2017}
{(11 months)}
{Center for Research and Advanced Studies of the IPN (CINVESTAV)}
{Irapuato, Mexico}
{{\normalsize\emph{Research intern in Microcosms Community Dynamics}}}
{\emph{Advisor:} Prof Gabriela Olmedo-Álvarez }%\\
%\emph{Description:} computational analysis of a 2-year dataset capturing the community dynamics of oligotrophic bacteria in a microcosms setting resembling their environment.}
%------------------------------------------------
\entry
{2013}
{(2 months)}
{CINVESTAV}
{Irapuato, Mexico}
{{\normalsize\emph{Summer undergraduate research in Microbial Ecology modeling}}}
{\emph{Advisors:} Prof Gabriela Olmedo-Álvarez \& Prof Moisés Santillán }%\\
%\emph{Related activities:} computational modeling of assembling bacterial communities in an oligrotrophic environment and experimental measurement of growth and death rates.}
%------------------------------------------------
\entry
{2012}
{(6 months)}
{* Polytechnic University of Madrid}
{Madrid, Spain}
{{\normalsize\emph{Magalhães/SMILE Exchange Semester}}}
{\emph{Coursework:} advanced numerical methods, processes optimization, machine learning and neural networks, biomaterials \& nanotechnology.}
%------------------------------------------------
\entry
{2012}
{(2 months)}
{CINVESTAV}
{Monterrey, Mexico}
{{\normalsize\emph{Summer undergraduate research in Biological Circuitry modeling}}}
{\emph{Advisor:} Prof Moisés Santillán Zerón }%\\ \emph{Related activities:} mathematical modeling of a synthetic genetic circuit generating syncronized oscillations of a bacterial population, and study of the criticality of its dynamics.}
%------------------------------------------------
\entry
{2011}
{(2 months)}
{CINVESTAV}
{Irapuato, Mexico}
{{\normalsize\emph{Summer undergraduate research in Synthetic Biology}}}
{\emph{Advisors:} Prof Agustino Martínez Antonio \& Prof Alexander DeLuna}%\\
%\emph{Related activities:} design, construction and modeling of a synthetic genetic circuit in \textit{E. coli} (\textit{AND} logic gate) generating different phenotypic responses to different substrates.}
%------------------------------------------------
\end{entrylist}
\begin{entrylist}
\entry
{2009 -- 2013}
{}
{* National Polytechnic Institute (IPN)}
{Silao, Mexico}
{{\normalsize\emph{BSc {\normalfont Biotechnological Engineering}} \\} \emph{Related coursework:} microbiology, biochemistry, molecular biology, bioengineering, differential equations, vectorial calculus, probability and statistics, basic physics, thermodynamics, unit operations \& industrial process design.}
{\emph{Thesis:} The role of the antagonism in the self-assembly of bacterial communities \\ \emph{Advisors:} Prof Gabriela Olmedo-Álvarez \& Prof Moisés Santillán (CINVESTAV) \\ \emph{Summary:} using a cellular automaton we investigated the role of experimentally quantified antagonism to preserve the biodiversity in \textit{Bacillus} communities from an oligotrophic pond in Cuatro Cienegas - Mexico. This thesis was done from 2014 to 2015.}
%------------------------------------------------
\end{entrylist}

%----------------------------------------------------------------------------------------
%	PUBLICATIONS SECTION
%----------------------------------------------------------------------------------------

\section{publications \& preprints}

\begin{enumerate}

\item  Gao Y, Pichugin Y, Traulsen A \& \textbf{Zapién-Campos R} (2025). Evolution of irreversible differentiation under stage-dependent cell differentiation. \textit{Scientific Reports}. \href{https://doi.org/10.1038/s41598-025-91838-8}{doi: 10.1038/s41598-025-91838-8}

\item \textbf{Zapién-Campos R}, Bansept F, \& Traulsen A. (2024). Stochastic models allow improved inference of microbiome interactions from time series data. \textit{PLoS Biology}. \href{https://doi.org/10.1371/journal.pbio.3002913}{doi: 10.1371/journal.pbio.3002913}

\item  \textbf{Zapién-Campos R}, Bansept F, Sieber M, \& Traulsen A. (2022). On the effect of inheritance of microbes in commensal microbiomes. \textit{BMC Ecology and Evolution}, 22(1), 1-13. \href{https://doi.org/10.1186/s12862-022-02029-2}{doi: 10.1186/s12862-022-02029-2}

\item  \textbf{Zapién-Campos R}, Sieber M, \& Traulsen A. (2022). The effect of microbial selection on the occurrence-abundance patterns of microbiomes. \textit{Interface: Journal of the Royal Society}, 19(187): 20210717. \href{https://doi.org/10.1098/rsif.2021.0717}{doi: 10.1098/rsif.2021.0717}

\item \textbf{Zapién-Campos R}, Sieber M, \& Traulsen A. (2020). Stochastic colonization of hosts with a finite lifespan can drive individual host microbes out of equilibrium. \textit{PLoS computational biology}, 16(11), e1008392. \href{https://doi.org/10.1371/journal.pcbi.1008392}{doi: 10.1371/journal.pcbi.1008392}

\item \textbf{Zapién-Campos R}, Olmedo-Álvarez G \& Santillán M. (2015) Antagonistic interactions are sufficient to explain self-assemblage of bacterial communities in a homogeneous environment: a computational modeling approach. \textit{Frontiers in Microbiology}. 6:489, 1--9 pp. \href{http://dx.doi.org/10.3389/fmicb.2015.00489}{doi: 10.3389/fmicb.2015.00489}

\end{enumerate}

%
%%\printbibsection{article}{articles in peer-reviewed journals} % Print all articles from the bibliography
%%\printbibsection{book}{books} % Print all books from the bibliography
%
%%\begin{refsection} % This is a custom heading for those references marked as "inproceedings" but not containing "keyword=france"
%%\nocite{*}
%%\printbibliography[sorting=chronological, type=inproceedings, title={international peer-reviewed conferences/proceedings}, notkeyword={france}, heading=subbibliography]
%%\end{refsection}
%
%%\begin{refsection} % This is a custom heading for those references marked as "inproceedings" and containing "keyword=france"
%%\nocite{*}
%%\printbibliography[sorting=chronological, type=inproceedings, title={local peer-reviewed conferences/proceedings}, keyword={france}, heading=subbibliography]
%%\end{refsection}
%
%%\printbibsection{misc}{other publications} % Print all miscellaneous entries from the bibliography
%
%%\printbibsection{report}{research reports} % Print all research reports from the bibliography

%----------------------------------------------------------------------------------------
%	FUNDING AND AWARDS SECTION
%----------------------------------------------------------------------------------------

\section{3rd party funding \& fellowships}

\begin{entrylist}
%------------------------------------------------
\entry
{2024 -- 2026}
{Fellowship}
{Marie Skłodowska-Curie Actions (European Commission \& UKRI)}
{London, UK}
{{\normalsize  236,748.48 EUR}}
{Awarded to applicants (15.8\% in 2023) holding a PhD to carry out their postdoctoral research, acquire new skills and develop their career.}

%------------------------------------------------
\entry
{2015 -- 2016}
{Fellowship}
{National Council of Science and Technology (CONACYT)}
{London, UK}
{{\normalsize 300,000 MXN + 10,560 GBP}}
{Awarded to top ranked Mexican applicants to pursue graduate STEM studies abroad.}
%------------------------------------------------
\entry
{2011 - 2013}
{Scholarship}
{Interinstitutional Program for the Strengthening of Research and Graduate Studies of the Pacific (Programa Delfín)}
{Irapuato, Monterrey \& Irapuato, Mexico}
{{\normalsize 13,000 MXN (2013), 13,000 MXN (2012) \& 13,000 MXN (2011)}}
{Awarded to top ranked BSc applicants for a 2-months intership in a lab of their choice.}
%------------------------------------------------
\end{entrylist}

%----------------------------------------------------------------------------------------
%	TEACHING SECTION
%----------------------------------------------------------------------------------------

\section{teaching \& supervision}

\begin{entrylist}
\entry
{2025}
{(2 lectures)}
{University College London}
{London, UK}
{{\normalsize\emph{Data Science Methods in Biology.} Lecturer}}
{\emph{Related activities:} lectures, practicals, and grading of year 2 BSc Biological Sciences students.
}
\end{entrylist}
%------------------------------------------------
\begin{entrylist}
\entry
{2023}
{(6 weeks)}
{Max Planck Institute for Evolutionary Biology}
{Plön, Germany}
{{\normalsize\emph{Modelling and inference from microbiome data.} Supervisor}}
{\emph{Related activities:} supervision of Yan Giencke (MSc) in Prof Arne Traulsen's group.}
\end{entrylist}
\begin{entrylist}
%------------------------------------------------
\entry
{2021}
{(4 lectures)}
{University of Lübeck}
{Lübeck, Germany}
{{\normalsize\emph{Mathematical biology in action during the covid-19 pandemic.} Lecturer}}
{\emph{Related activities:} lectures and review of assignments of students in MSc Mathematics.}
%------------------------------------------------
\entry
{2021}
{(12 weeks)}
{Max Planck Institute for Evolutionary Biology}
{Online}
{{\normalsize\emph{Mobility data analysis during the covid-19 pandemic in Germany.} Supervisor}}
{\emph{Related activities:} supervision of Carissa Reid (MSc) in Prof Arne Traulsen's group.}
%------------------------------------------------
\entry
{2017}
{(6 weeks)}
{CINVESTAV}
{Irapuato, Mexico}
{{\normalsize\emph{Statistics in R workshop.} Instructor}}
{\emph{Related activities:} teaching a group of graduate students and staff Statistics and its implementation in R, including: Exploratory Data Analysis, Statistics and R coding.}
%------------------------------------------------
\entry
{2014}
{(6 weeks)}
{CINVESTAV}
{Irapuato, Mexico}
{{\normalsize\emph{Programming Club: Introduction to Python and R.} Instructor}}
{\emph{Related activities:} leading a group of undergrad and grad students to learn programming and scripting skills in Python and R analysing their own data.}
%------------------------------------------------
\end{entrylist}

%----------------------------------------------------------------------------------------
%	PEER-REVIEW SECTION
%----------------------------------------------------------------------------------------
\section{peer-review for scientific journals}
BMC Biology (1x),  Journal of Nonlinear Science (1x), Molecular Ecology (1x), and PLoS Computational Biology (2x).

%----------------------------------------------------------------------------------------
%	ORGANIZATIONAL SECTION
%----------------------------------------------------------------------------------------
\section{organization of scientific events}

\begin{entrylist}
%------------------------------------------------
\entry
{2022}
{}
{Max Planck Institute for Evolutionary Biology}
{Plön, Germany}
{{\normalsize\emph{Mathematical modelling of microbiomes.} Co-organizer}}
{\emph{Related activities:} invitation of international speakers, internal and external communication, preparation of a scientific program, chairing and management of the event.}
%------------------------------------------------
\entry
{2019}
{}
{Max Planck Institute for Evolutionary Biology}
{Oeversee, Germany}
{{\normalsize\emph{Scientific retreat of the doctoral school (IMPRS).} Co-organizer}}
{\emph{Related activities:} preparation of a scientific and career development program, invitation of international speakers, internal and external communication, chairing and management of the event.}
%------------------------------------------------
\entry
{2019}
{}
{Max Planck Institute for Evolutionary Biology}
{Plön, Germany}
{{\normalsize\emph{Internal symposium (Aquavit).} Co-organizer}}
{\emph{Related activities:} preparation of a scientific program, internal communication, chairing and management of the event.}
%------------------------------------------------
\end{entrylist}

%----------------------------------------------------------------------------------------
%	OUTREACH SECTION
%----------------------------------------------------------------------------------------

\section{scientific outreach}

\begin{entrylist}
%\entry
%{2025}
%{}
%{Pint of Science}
%{London, UK}
%{{\normalsize\emph{Pint of Science 2025 - Planet Earth team @ UCL.} Co-organizer}}
%{\emph{Related activities:} prepared the event with a team, identified and booked the venue, designed a scientific program, invited speakers, promoted the event, maintained a website and hosted the event three evenings.}
%%------------------------------------------------
\entry
{2020 -- 2021}
{}
{Max Planck Institute for Evolutionary Biology}
{Plön, Germany}
{{\normalsize\emph{Communication of the covid-19 pandemic.} Team member}}
{\emph{Related activities:} produced interactive online plots to communicate the current state of the pandemic in  Germany and Schleswig-Holstein to the public and government. I also developed a timelapse map to show the spread of cases across Germany.\\
\** We received a letter of acknowledgement from the state governor Daniel Günther.}
%------------------------------------------------
\end{entrylist}
\begin{entrylist}
\entry
{2010 -- 2015}
{}
{National Polytechnic Institute}
{Silao, Mexico}
{{\normalsize\emph{UPIIG’s Science Club.} Co-organizer}}
{\emph{Related activities:} study and discussion of STEM topics, participation in scholar events, and organization of seminars lead by students and local scientists about their research expertise and the experience of being and becoming a scientist.}
%------------------------------------------------
\end{entrylist}

%----------------------------------------------------------------------------------------
%	CONFERENCES AND WORKSHOPS SECTION
%----------------------------------------------------------------------------------------

\section{talks, conferences \& workshops}

\textbf{Invited Talk}

\begin{entrylist}
%\entry
%{2025}
%{}
%{CENTURI - University of Marseille}
%{Marseille, France}
%{{\normalsize\emph{Seminar series}}}
%{\vspace{-3mm}}
%%------------------------------------------------
\entry
{2023}
{}
{DFG Research Training Group for Translational Evolutionary Research}
{Plön, Germany}
{{\normalsize\emph{Seminar series}}}
{\vspace{-3mm}}
%------------------------------------------------
\end{entrylist}
\begin{entrylist}
\entry
{2022}
{}
{Mexican Society of Mathematics}
{Online}
{{\normalsize\emph{National Congress 2022}}}
{\vspace{-3mm}}
%------------------------------------------------
\end{entrylist}

\textbf{Contributed Talk}

\begin{entrylist}
\entry
{2025}
{}
{Gordon Research Conference}
{Ventura, USA}
{{\normalsize\emph{Stochastic Physics in Biology}}}
{\vspace{-3mm}}
%------------------------------------------------
\entry
{2024}
{}%(1 week)
{Max Planck Institute for Evolutionary Biology}
{Plön, Germany}
{{\normalsize\emph{Developments in Evolutionary Theory}}}
{\vspace{-3mm}}
%------------------------------------------------
\entry
{2024}
{}%(1 week)
{University of Warwick}
{Warwick, UK}
{{\normalsize\emph{Discussions on Microbial Ecology}}}
{\vspace{-3mm}}
%------------------------------------------------
\entry
{2022}
{}%(1 week)
{Max Planck Institute for Evolutionary Biology}
{Plön, Germany}
{{\normalsize\emph{Physical and Chemical Determinants of Biological Evolution}}}
{\vspace{-3mm}}
%------------------------------------------------
\entry
{2022}
{(3 weeks)}
{International Centre for Theoretical Physics}
{Trieste, Italy}
{{\normalsize\emph{Eco-evolutionary Dynamics of Microbial Communities}}}
{\vspace{-3mm}}
%------------------------------------------------
\end{entrylist}
\begin{entrylist}
\entry
{2022}
{}%(1 week)
{Max Planck Institute for Evolutionary Biology}
{Plön, Germany}
{{\normalsize\emph{Meetings on Microbial Ecology and Evolution}}}
{\vspace{-3mm}}
%------------------------------------------------
\entry
{2019}
{}%(1 week)
{Max Planck Institute for Evolutionary Biology}
{Plön, Germany}
{{\normalsize\emph{Evolution of Interacting Populations}}}
{\vspace{-3mm}}
%------------------------------------------------
\entry
{2016}
{}%(1 week)
{Society of Mexican Students in the United Kingdom}
{Edinburgh, UK}
{{\normalsize\emph{XIV Symposium of Mexican Studies and Students}}}
{\vspace{-3mm}}
%------------------------------------------------
\entry
{2015}
{}%(1 week)
{Mexican Scientific Society of Ecology}
{San Luis Potosi, Mexico}
{{\normalsize\emph{V Mexican Congress of Ecology}}}
{\vspace{-3mm}}
%------------------------------------------------
\end{entrylist}

\textbf{Poster}

\begin{entrylist}
\entry
{2025}
{}
{The Francis Crick Institute}
{London, UK}
{{\normalsize\emph{INTERPHACE symposium}}}
{\vspace{-3mm}}
%------------------------------------------------
\entry
{2025}
{}
{Gordon Research Seminar}
{Ventura, USA}
{{\normalsize\emph{Stochastic Physics in Biology}}}
{\vspace{-3mm}}
%------------------------------------------------
\entry
{2024}
{}
{The Francis Crick Institute}
{London, UK}
{{\normalsize\emph{London Mathematical Biology Conference}}}
{\vspace{-3mm}}
%------------------------------------------------
\entry
{2020 \& '21}
{}%(1 week)
{Society for Mathematical Biology}
{Online}
{{\normalsize\emph{eSMB 2020 and SMB 2021}}}
{\vspace{-3mm}}
%------------------------------------------------
\end{entrylist}
\begin{entrylist}
\entry
{2019 to '21}
{}%(1 week)
{Max Planck Institute for Evolutionary Biology}
{Plön, Germany and Online}
{{\normalsize\emph{Internal symposium: Aquavit.} (Won best poster in 2019 \& 2020)}}
{\vspace{-3mm}}
%------------------------------------------------
\entry
{2019}
{}%(1 week)
{Gordon Research Conference \& Gordon Research Seminar}
{West Dover, USA}
{{\normalsize\emph{Animal-Microbe Symbioses}}}
{\vspace{-3mm}}
%------------------------------------------------
\end{entrylist}
\begin{entrylist}
\entry
{2017}
{}%(1 week)
{Mexican Association of Microbiology}
{Guadalajara, Mexico}
{{\normalsize\emph{XL National Congress of Microbiology}}}
{\vspace{-3mm}}
%------------------------------------------------
\end{entrylist}

\textbf{Participant}

\begin{entrylist}
\entry
{2013}
{(1 week)}
{Center for Research in Mathematics}
{Guanajuato, Mexico}
{{\normalsize\emph{Summer school on molecular dynamics and quantum chemistry}}}
{\vspace{-3mm}}
%------------------------------------------------
\entry
{2012}
{(1 week)}
{Center for Research in Mathematics}
{Guanajuato, Mexico}
{{\normalsize\emph{Summer school on mathematical modeling of biological systems}}}
{\vspace{-3mm}}
%------------------------------------------------
\entry
{2012}
{(1 week)}
{Iberoamerican Society of Bioinformatics}
{Cuernavaca, Mexico}
{{\normalsize\emph{International workshops in bioinformatics 2012}}}
{\vspace{-3mm}}
%------------------------------------------------
%\entry
%{2011 (1 wk)}
%{National Polytechnic Institute}
%{Silao, Mexico}
%{{\normalsize\emph{Workshop on computational chemistry}}}
%{\vspace{-3mm}}
%------------------------------------------------
%\entry
%{2010 (2 wks)}
%{National Polytechnic Institute}
%{Silao, Mexico}
%{{\normalsize\emph{Workshop on \textit{in vitro} vegetal culture techniques}}}
%{\vspace{-3mm}}
%------------------------------------------------
\end{entrylist}

%\pagebreak

\end{document}